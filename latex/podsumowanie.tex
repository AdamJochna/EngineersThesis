\chapter{Podsumowanie}
\thispagestyle{chapterBeginStyle}

Wszystkie etapy pracy inżynierskiej zostały wykonane w pełnym zakresie. Szczegółowo opisano zbiór danych oraz przedstawiono możliwe sposoby jego przetwarzania (rasteryzatory). Znaleziono architekturę sieci neuronowej, która spełnia zadanie predykcji zachowań uczestników ruchu drogowego z bardzo wysoką skutecznością. W pracy zostały opisane eksperymenty, które doprowadziły do uzyskanej architektury. Został omówiony i zaimplementowany nietypowy algorytm agregacji predykcji modeli, który jak się okazało zwiększa skuteczność systemu do poziomu, który nie jest osiągalny za pomocą pojedynczych sieci neuronowych uzyskanych w procesach trenowania.

\vspace{1em}

Należy zaznaczyć, że problem predykcji zachowań uczestników ruchu drogowego jest problemem otwartym, nie istnieje obecnie system, który przewiduje pozycje bezbłędnie. Usprawnienia w rozwiązaniach tego problemu mają na celu nie tylko rozwój nauki, przede wszystkim mają na celu zwiększenie bezpieczeństwa ludzi w środowisku, które w niedalekiej przyszłości może być w dużej mierze zdominowane przez pojazdy autonomiczne. Analiza modeli predykcyjnych opisanych w tej pracy, robiona była z nadzieją, że zwiększy aktualny stan wiedzy na temat systemów predykcyjnych w dziedzinie, która rozwija się bardzo dynamicznie i stanowi bezpośredni czynnik ludzkiego bezpieczeństwa.