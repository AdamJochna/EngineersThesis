\chapter*{Wstęp}
\addcontentsline{toc}{chapter}{Wstęp}

\thispagestyle{chapterBeginStyle}

\section*{Cel pracy}

Celem pracy inżynierskiej jest skonstruowanie modelu pozwalającego przewidzieć położenie pojazdu uczestniczącego w ruchu drogowym. Formalnie problem ten polega na znalezieniu funkcji $f$, która przekształca dane wejściowe na predykcje. Określenie funkcja $f$ i model predykcyjny jest w kontekście tej pracy tożsame.

\section*{Zakres pracy}

\noindent Zakres pracy obejmuje następujące zagadnienia:

\begin{itemize}
    \setlength{\itemsep}{1pt}
    \setlength{\parskip}{0pt}
    \setlength{\parsep}{0pt}
    \item Opis zbioru dotyczącego ruchu drogowego
    \item Wybór optymalnego sposobu przetwarzanie zbioru
    \item Zaproponowanie architektur modeli predykcyjnych i zaprogramowanie ich
    \item Wytrenowanie kilku sieci neuronowych
    \item Wybór optymalnego modelu predykcyjnego ruchu agentów
    \item Porównanie modeli opartych o głębokie sieci neuronowe z prostymi modelami kinetycznymi
    \item Porównanie metod łączenia wyników wielu modeli predykcyjnych
\end{itemize}

\section*{Zawartość pracy}

\noindent Rozdziały pracy wraz z opisem zawartości:
\begin{enumerate}[label*=\arabic*.]
    \setlength{\itemsep}{1pt}
    \setlength{\parskip}{0pt}
    \setlength{\parsep}{0pt}
    \item \textbf{Analiza problemu} - Opis problemu, klasyfikacja typu problemu, opis struktury zbioru oraz sposobów na wczytywanie jego elementów.
    \item \textbf{Rasteryzacja sceny} - Opis możliwych sposobów uzyskania informacji na temat scen w formie obrazu. Opis rasteryzatora z biblioteki \texttt{l5kit} oraz rasteryzatora \texttt{CenterLines}. Pokazanie różnic tych rasteryzatorów. Opis algorytmów, jakie są wykorzystywane przez te rasteryzatory.
    \item \textbf{Funkcja kosztu} - Opis funkcji kosztu, pokazanie jakie są jej składowe oraz wizualizacja wartości tej funkcji dla różnych danych wejściowych.
    \item \textbf{Naiwna metoda rozwiązania problemu} - Przedstawienie sposobu na przewidywanie ruchu agenta \texttt{EGO} za pomocą prostego modelu wykorzystującego pewne proste założenia na temat ruchu agentów.
    \item \textbf{Głębokie sieci neuronowe} - Przedstawienie architektury sieci neuronowych pozwalających przewidywać ruch agentów, z użyciem obrazu \texttt{BEV} (widok z góry na scenę).
    \item \textbf{Proces trenowania sieci neuronowych} - Opis procesu trenowania oraz tego jak efektywnie trenować kilka sieci równolegle. Opis strategii walidacji procesu trenowania.
    \item \textbf{Agregacja modeli} - Opis sposobów jakimi można połączyć wyniki kilku modeli. Opis tego w jaki sposób takie podejście wpływa na jakość i szybkość predykcji.
    \item \textbf{Opis ostatecznego rozwiązania} - Opis najlepszego sposobu na uzyskanie predykcji. Opis parametrów sieci oraz analiza przypadków w których sieć popełnia błędy.
    \item \textbf{Determinizm procesu trenowania} - Opis sposobu uzyskania powtarzalności procesu trenowania.
\end{enumerate}